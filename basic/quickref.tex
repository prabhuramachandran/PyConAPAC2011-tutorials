\documentclass{article}
\usepackage{graphicx}  
\usepackage[landscape]{geometry}
\usepackage[pdftex]{color}
\usepackage{url}
\usepackage{multicol}
\usepackage{amsmath}
\usepackage{amsfonts}
\newcommand{\ex}{\color{blue}}
\pagestyle{empty}
\advance\topmargin-.9in
\advance\textheight2in
\advance\textwidth3.0in
\advance\oddsidemargin-1.45in
\advance\evensidemargin-1.45in
\parindent0pt
\parskip2pt

\newcommand{\hr}{\centerline{\rule{3.5in}{1pt}}}
\newcommand{\skipin}{\hspace*{12pt}}

\usepackage{color}
\definecolor{darkgreen}{rgb}{0,0.5,0}

\usepackage{listings}
\lstset{language=Python,
    basicstyle=\ttfamily\bfseries,
    commentstyle=\color{red}\itshape,
  stringstyle=\color{darkgreen},
  showstringspaces=false,
  keywordstyle=\color{blue}\bfseries}

\begin{document}
\begin{multicols*}{3}
\begin{center}
\textbf{Python for Scientific Computing}\\
\textbf{Quick Reference}\\
FOSSEE -- Dec 1, 2010\\
\end{center}
\vspace{-2ex}

%*********************************************
 \hr\textbf{Starting up}

To start \lstinline|ipython| with \lstinline|pylab|:\\
\lstinline|    $ ipython -pylab| %$

To exit: \lstinline|^D| (Ctrl-D)

%*********************************************
\hr\textbf{Plotting}

Creating a linear array:\\
{\ex \lstinline|    x = linspace(0, 2*pi, 50)|}

Plotting two variables:\\
{\ex \lstinline|    plot(x, sin(x))|}

Plotting two lists of equal length x, y:\\
{\ex \lstinline|    plot(x, y)|}

Plots with colors:\\
{\ex \lstinline|    plot(x, sin(x), 'b')|} gives a blue line

Line style and markers:\\
{\ex \lstinline|    plot(x, sin(x), '--')|} gives a dashed line\\
{\ex \lstinline|'.'|} --  a point marker, {\ex \lstinline|'o'|} -- a circle marker

Labels:\\
{\ex \lstinline|    xlabel('x')|} and {\ex \lstinline|ylabel('sin(x)')|}

Title (\lstinline|pylab| accepts \TeX~in any text expression):\\
{\ex \lstinline|    title(r'$\sigma$ vs. $\sin(\sigma)$')|} 

Legend:\\
Use standard placement:\\
{\ex \lstinline|    legend('sin(x)', loc=center)|}\\
Or explicitly specify location:\\
{\ex \lstinline|    legend(['sin(x)'], loc=(.8,.1))|}

Annotate:\\
{\ex \lstinline|    legend('annotation string', xy=(1.5, 1))|}

Saving figures:\\
{\ex \lstinline|    savefig('sinusoids.png')|}

Miscellaneous:\\
{\ex \lstinline|    clf()|} to clear the plot area\\
{\ex \lstinline|    close()|} to close the figure

%*********************************************
\textbf{Other plots}

Pie Charts:\\
{\ex \lstinline|    pie(science.values(), labels=science.keys())|},
where {\ex \lstinline|science|} is a dictionary.

%*********************************************
\hr\textbf{Saving and Running scripts}

{\ex \lstinline|    %hist|} returns history of commands used. \\
To save a set of lines, say 14-18, 20, 22, to \lstinline|sample.py| \\
{\ex \lstinline|    %save sample.py 14-18 20 22|}\\
To run \lstinline|sample.py| \\
{\ex \lstinline|    %run -i sample.py|}

%*********************************************
 \hr\textbf{Reading from files}

\lstinline|filename.txt| is a file with float data. 

Using a for loop: (reading line by line)
\vspace{-8pt}
\begin{lstlisting}
for line in open('filename.txt'):
    # some processing
\end{lstlisting}
\vspace{-8pt}
Using loadtxt: (reading all the data at once)\\
{\ex \lstinline|X = loadtxt('filename.txt')|} \\
X is an array with all the data from \lstinline|filename.txt|\\
{\ex \lstinline|X = loadtxt('filename.txt', delimiter=';')|} \\
when ';' delmits the columns of data

%*********************************************
\hr\textbf{Statistical operations}

{\ex \lstinline|mean|, \lstinline|median|, \lstinline|std|}

%*********************************************
\hr\textbf{Array Creation}

{\ex \lstinline|C = array([[11,12,13], [21,22,23], [31,32,33]])|}\\
{\ex \lstinline|C.shape|} shape--- rows \& cols\\
{\ex \lstinline|C.dtype|} data type\\
{\ex \lstinline|B = ones_like(C)|} array of ones; same shape, dtype as C\\
\skipin similarly \lstinline|zeros_like, empty_like| \\
{\ex \lstinline|A = ones((3,2))|} array of ones of shape (3,2)\\
\skipin similarly \lstinline|zeros, empty|\\
{\ex \lstinline|I = identity(3)|} identity matrix of size 3x3

%*********************************************
\hr\textbf{Accessing \& Changing elements}

{\ex \lstinline|C[1, 2]|} gets third element of second row\\
\skipin \textbf{Note:} Indexing starts from 0. \\
{\ex \lstinline|C[1]|} gets the second row \\
{\ex \lstinline|C[1,:]|} same as above (`:' implies all columns)\\
{\ex \lstinline|C[:,1]|} gets the second column (`:' implies all rows)\\
{\ex \lstinline|C[0:2,:]|} or {\ex \lstinline|C[:2,:]|} gets $1^{st}, 2^{nd}$ rows; all cols\\
{\ex \lstinline|C[1:3,:]|} or {\ex \lstinline|C[1:,:]|} gets $2^{nd}, 3^{rd}$
rows; all cols\\
{\ex \lstinline|C[0:3:2,:]|} or {\ex \lstinline|C[::2,:]|} gets $1^{st}, 3^{rd}$
rows; all cols
 
%*********************************************
\hr\textbf{Matrix Operations}

For a Matrix A and B of equal shapes:\\
{\ex \lstinline|A.T|} transpose\\
{\ex \lstinline|sum(A)|} sum of all elements\\
{\ex \lstinline|A+B|} addition\\
{\ex \lstinline|A*B|} element wise product\\
{\ex \lstinline|dot(A, B)|} Matrix multiplication\\
{\ex \lstinline|inv(A)|} inverse, {\ex \lstinline|det(A)|} determinant\\
{\ex \lstinline|eig(A)|} eigen values and vectors\\
{\ex \lstinline|norm(A)|} norm\\
{\ex \lstinline|svd(A)|} singular value decomposition

%*********************************************
\hr\textbf{Solving Linear Equations}

{\ex \lstinline|A = array([[3,2,-1], [2,-2,4], [-1, 0.5, -1]])|}\\
\skipin coefficient array\\
{\ex \lstinline|b = array([1, -2, 0])|} constant array\\
{\ex \lstinline|x = solve(A, b)|} the required solution

Checking the solution:\\
{\ex \lstinline|Ax = dot(A,x)|} matrix multiplication of A and x\\
{\ex \lstinline|allclose(Ax, b)|} check the closeness of Ax, b

%*********************************************
\hr\textbf{Roots of Polynomials}

{\ex \lstinline|coeffs = [1, 6, 13]|} coefficients in descending order\\ 
{\ex \lstinline|roots(coeffs)|} returns complex roots of the polynomial

%*********************************************
\hr\textbf{Roots of non-linear equations}

{\ex \lstinline|from scipy.optimize import fsolve|}\\
\skipin \lstinline|fsolve| is not in \lstinline|pylab|\\
\skipin we import from \lstinline|scipy.optimize| \\ 
We wish to find the roots of $f(x)=sin(x)+cos(x)^2$ 
\vspace{-8pt}
\begin{lstlisting}
def f(x):
    return sin(x)+cos(x)**2
\end{lstlisting}
\vspace{-8pt}
{\ex \lstinline|fsolve(f, 0)|} \\
\skipin arguments are function name and initial estimate

%*********************************************
\hr\textbf{ODE}

To solve the ODE below:\\
$\frac{dy}{dt} = ky(L-y)$, L = 25000, k = 0.00003, y(0) = 250\\
\vspace{-8pt}
\begin{lstlisting}
def f(y, t):
    k, L = 0.00003, 25000
    return k*y*(L-y)
\end{lstlisting}
\vspace{-8pt}
{\ex \lstinline|t = linspace(0, 12, 60)|} time over which to solve ODE\\
{\ex \lstinline|y0 = 250|} initial conditions\\
{\ex \lstinline|from scipy.integrate import odeint|}\\
{\ex \lstinline|y = odeint(f, y0, t)|}

\hr\textbf{FFT}

{\ex \lstinline|t = linspace(0, 2*pi, 500)|}\\
{\ex \lstinline|y = sin(4*pi*t)|} a sinusoidal signal\\
{\ex \lstinline|f = fft(y)|}\\
{\ex \lstinline|freq = fftfreq(500, t[1] - t[0])|}\\
{\ex \lstinline|plot(freq[:250], abs(f)[:250])|}\\

\end{multicols*}

\end{document}
