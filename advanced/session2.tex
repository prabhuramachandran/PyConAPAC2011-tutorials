%%%%%%%%%%%%%%%%%%%%%%%%%%%%%%%%%%%%%%%%%%%%%%%%%%%%%%%%%%%%%%%%%%%%%%%%%%%%%%%%
%Tutorial slides on Python.
%
% Author: FOSSEE 
% Copyright (c) 2009-2011, FOSSEE, IIT Bombay
%%%%%%%%%%%%%%%%%%%%%%%%%%%%%%%%%%%%%%%%%%%%%%%%%%%%%%%%%%%%%%%%%%%%%%%%%%%%%%%%

\documentclass[14pt,compress]{beamer}

% Modified from: generic-ornate-15min-45min.de.tex
\mode<presentation>
{
  \usetheme{Warsaw}
  \useoutertheme{infolines}
  \setbeamercovered{transparent}
}

\usepackage[english]{babel}
\usepackage[latin1]{inputenc}
%\usepackage{times}
\usepackage[T1]{fontenc}
\usepackage{pgf}

% Taken from Fernando's slides.
\usepackage{ae,aecompl}
\usepackage{mathpazo,courier,euler}
\usepackage[scaled=.95]{helvet}

\definecolor{darkgreen}{rgb}{0,0.5,0}

\usepackage{listings}
\lstset{language=Python,
    basicstyle=\ttfamily\bfseries,
    commentstyle=\color{red}\itshape,
  stringstyle=\color{darkgreen},
  showstringspaces=false,
  keywordstyle=\color{blue}\bfseries}

%%%%%%%%%%%%%%%%%%%%%%%%%%%%%%%%%%%%%%%%%%%%%%%%%%%%%%%%%%%%%%%%%%%%%%
% Macros
\setbeamercolor{emphbar}{bg=blue!20, fg=black}
\newcommand{\emphbar}[1]
{\begin{beamercolorbox}[rounded=true]{emphbar} 
      {#1}
 \end{beamercolorbox}
}
\newcounter{time}
\setcounter{time}{0}
\newcommand{\inctime}[1]{\addtocounter{time}{#1}{\tiny \thetime\ m}}

\newcommand{\typ}[1]{\lstinline{#1}}

\newcommand{\kwrd}[1]{ \texttt{\textbf{\color{blue}{#1}}}  }

%%% This is from Fernando's setup.
% \usepackage{color}
% \definecolor{orange}{cmyk}{0,0.4,0.8,0.2}
% % Use and configure listings package for nicely formatted code
% \usepackage{listings}
% \lstset{
%    language=Python,
%    basicstyle=\small\ttfamily,
%    commentstyle=\ttfamily\color{blue},
%    stringstyle=\ttfamily\color{orange},
%    showstringspaces=false,
%    breaklines=true,
%    postbreak = \space\dots
% }

%%%%%%%%%%%%%%%%%%%%%%%%%%%%%%%%%%%%%%%%%%%%%%%%%%%%%%%%%%%%%%%%%%%%%%
% Title page
\title[Advanced Sci Comp.]{Advanced Scientific Computing with
Python}
\subtitle{Virtualenv, Cython }

\author[FOSSEE] {FOSSEE}

\institute[IIT Bombay] {Department of Aerospace Engineering\\IIT Bombay}
\date[] {PyCon Asia-Pacific,\\
Singapore\\
June 9, 2011
}
%%%%%%%%%%%%%%%%%%%%%%%%%%%%%%%%%%%%%%%%%%%%%%%%%%%%%%%%%%%%%%%%%%%%%%

%\pgfdeclareimage[height=0.75cm]{iitmlogo}{iitmlogo}
%\logo{\pgfuseimage{iitmlogo}}


%% Delete this, if you do not want the table of contents to pop up at
%% the beginning of each subsection:
\AtBeginSubsection[]
{
  \begin{frame}<beamer>
    \frametitle{Outline}
    \tableofcontents[currentsection,currentsubsection]
  \end{frame}
}

\AtBeginSection[]
{
  \begin{frame}<beamer>
    \frametitle{Outline}
    \tableofcontents[currentsection,currentsubsection]
  \end{frame}
}

% If you wish to uncover everything in a step-wise fashion, uncomment
% the following command: 
%\beamerdefaultoverlayspecification{<+->}

%%\includeonlyframes{current,current1,current2,current3,current4,current5,current6}

%%%%%%%%%%%%%%%%%%%%%%%%%%%%%%%%%%%%%%%%%%%%%%%%%%%%%%%%%%%%%%%%%%%%%%
% DOCUMENT STARTS
\begin{document}

\begin{frame}
  \maketitle
\end{frame}

%% \begin{frame}
%%   \frametitle{Outline}
%%   \tableofcontents
%%   % You might wish to add the option [pausesections]
%% \end{frame}

\section{Virtualenv}


\begin{frame}
  \frametitle{Motivation}
  \begin{itemize}

      \item Need to install Python packages

      \item No root access

      \item Don't want to mess up system

      \item Create ``isolated'' environment

      \item PyPI
  
  \end{itemize}

\end{frame}

\begin{frame}[fragile]
  \frametitle{Virtualenv}

  \begin{itemize}
    \item  \url{www.virtualenv.org} 

    \item \url{pypi.python.org/pypi/virtualenv}

    \item Either install virtualenv 
        
    \item Oe use the \typ{virtualenv.py}

  \end{itemize}

  \begin{lstlisting}
$ easy_install virtualenv
  \end{lstlisting}
Or download tarball and
  \begin{lstlisting}
$ python setup.py install \
>  [--prefix=/usr/local]
  \end{lstlisting}

\end{frame}

\begin{frame}[fragile]
  \frametitle{Environments}
Creation:

  \begin{lstlisting}
$ python virtualenv.py ENV
  \end{lstlisting}

  \begin{itemize}
      \item \typ{ENV} is a directory name of your choice
      \item Self-contained universe
      \item By default (optional) inherits system packages
      \item Look in \typ{ENV/bin} (or \typ{ENV\\Scripts})

      \item Look in \typ{ENV/lib} (or \typ{ENV\\Lib})
  \end{itemize}

\end{frame}

\begin{frame}[fragile]
  \frametitle{Activation}

  \begin{lstlisting}
# Activation (Linux/Mac OSX)
$ source ENV/bin/activate

$ ENV\Scripts\activate.bat

# Deactivate
(ENV)$ deactivate

$ ENV\Scripts\deactivate.bat
  \end{lstlisting}
  \vspace*{0.1in}
  I prefer to activate in \typ{.bash_profile}
\end{frame}

\begin{frame}[fragile]
  \frametitle{Usage}
  \begin{lstlisting}
(ENV)$ python # this is ENV/bin/python

(ENV)$ easy_install my_fav_package
(ENV)$ pip install my_fav_package
# All above install into ENV.

(ENV)$ python setup.py install
# Also installs into ENV

  \end{lstlisting}

\end{frame}

\begin{frame}[fragile]
  \frametitle{Usage}
  \begin{lstlisting}
(ENV)$ deactivate
$ python virtualenv.py ANOTHER
(ANOTHER)$ source ANOTHER/bin/activate
(ANOTHER)$ pip install PKG
# installs in ANOTHER

(ANOTHER)$ deactivate
$ source ENV/bin/activate

# To remove ANOTHER
$ rm -rf ANOTHER
\end{lstlisting}

That is it! Easy, and very convenient
\end{frame}



\section{Cython}


\end{document}

