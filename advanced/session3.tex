%%%%%%%%%%%%%%%%%%%%%%%%%%%%%%%%%%%%%%%%%%%%%%%%%%%%%%%%%%%%%%%%%%%%%%%%%%%%%%%%
%Tutorial slides on Python.
%
% Author: FOSSEE 
% Copyright (c) 2009-2011, FOSSEE, IIT Bombay
%%%%%%%%%%%%%%%%%%%%%%%%%%%%%%%%%%%%%%%%%%%%%%%%%%%%%%%%%%%%%%%%%%%%%%%%%%%%%%%%

\documentclass[14pt,compress]{beamer}

% Modified from: generic-ornate-15min-45min.de.tex
\mode<presentation>
{
  \usetheme{Warsaw}
  \useoutertheme{infolines}
  \setbeamercovered{transparent}
}

\usepackage[english]{babel}
\usepackage[latin1]{inputenc}
%\usepackage{times}
\usepackage[T1]{fontenc}
\usepackage{pgf}

% Taken from Fernando's slides.
\usepackage{ae,aecompl}
\usepackage{mathpazo,courier,euler}
\usepackage[scaled=.95]{helvet}

\definecolor{darkgreen}{rgb}{0,0.5,0}

\usepackage{listings}
\lstset{language=Python,
    basicstyle=\ttfamily\bfseries,
    commentstyle=\color{red}\itshape,
  stringstyle=\color{darkgreen},
  showstringspaces=false,
  keywordstyle=\color{blue}\bfseries}

%%%%%%%%%%%%%%%%%%%%%%%%%%%%%%%%%%%%%%%%%%%%%%%%%%%%%%%%%%%%%%%%%%%%%%
% Macros
\setbeamercolor{emphbar}{bg=blue!20, fg=black}
\newcommand{\emphbar}[1]
{\begin{beamercolorbox}[rounded=true]{emphbar} 
      {#1}
 \end{beamercolorbox}
}
\newcounter{time}
\setcounter{time}{0}
\newcommand{\inctime}[1]{\addtocounter{time}{#1}{\tiny \thetime\ m}}

\newcommand{\typ}[1]{\lstinline{#1}}
\newcommand{\myemph}[1]{\structure{\emph{#1}}}
\newcommand{\PythonCode}[1]{\lstinline{#1}}

\newcommand{\kwrd}[1]{ \texttt{\textbf{\color{blue}{#1}}}  }

%%% This is from Fernando's setup.
% \usepackage{color}
% \definecolor{orange}{cmyk}{0,0.4,0.8,0.2}
% % Use and configure listings package for nicely formatted code
% \usepackage{listings}
% \lstset{
%    language=Python,
%    basicstyle=\small\ttfamily,
%    commentstyle=\ttfamily\color{blue},
%    stringstyle=\ttfamily\color{orange},
%    showstringspaces=false,
%    breaklines=true,
%    postbreak = \space\dots
% }

%%%%%%%%%%%%%%%%%%%%%%%%%%%%%%%%%%%%%%%%%%%%%%%%%%%%%%%%%%%%%%%%%%%%%%
% Title page
\title[Advanced Sci Comp.]{Advanced Scientific Computing with
Python}
\subtitle{Traits and Mayavi}

\author[FOSSEE] {FOSSEE}

\institute[IIT Bombay] {Department of Aerospace Engineering\\IIT Bombay}
\date[] {PyCon Asia-Pacific,\\
Singapore\\
June 9, 2011
}
%%%%%%%%%%%%%%%%%%%%%%%%%%%%%%%%%%%%%%%%%%%%%%%%%%%%%%%%%%%%%%%%%%%%%%

%\pgfdeclareimage[height=0.75cm]{iitmlogo}{iitmlogo}
%\logo{\pgfuseimage{iitmlogo}}


%% Delete this, if you do not want the table of contents to pop up at
%% the beginning of each subsection:
\AtBeginSubsection[]
{
  \begin{frame}<beamer>
    \frametitle{Outline}
    \tableofcontents[currentsection,currentsubsection]
  \end{frame}
}

\AtBeginSection[]
{
  \begin{frame}<beamer>
    \frametitle{Outline}
    \tableofcontents[currentsection,currentsubsection]
  \end{frame}
}

% If you wish to uncover everything in a step-wise fashion, uncomment
% the following command: 
%\beamerdefaultoverlayspecification{<+->}

%%\includeonlyframes{current,current1,current2,current3,current4,current5,current6}

%%%%%%%%%%%%%%%%%%%%%%%%%%%%%%%%%%%%%%%%%%%%%%%%%%%%%%%%%%%%%%%%%%%%%%
% DOCUMENT STARTS
\begin{document}

\begin{frame}
  \maketitle
\end{frame}

\section{Traits}

\begin{frame}
  \frametitle{Introduction to Traits}
  \begin{itemize}
    \item ETS: Enthought Tool Suite
    \item A \alert{trait} is a \alert{special} Python object attribute
        \vspace*{2em}

    \item \url{code.enthought.com/projects/traits}
    \item \url{github.enthought.com/traits/tutorials}

  \end{itemize}
\end{frame}

\begin{frame}
  \frametitle{Trait features}
  \begin{itemize}
    \item Initialization: default value
    \item Validation: strongly typed
    \item Delegation: value delegation
    \item Notification: events
    \item Visualization: MVC, automatic GUI!
  \end{itemize}
\end{frame}

\begin{frame}[fragile,plain]
  \frametitle{Traits Example}
\vspace*{-16pt}
\footnotesize
\begin{lstlisting}
from enthought.traits import Delegate, HasTraits, Int, Str

class Parent(HasTraits):
    # INITIALIZATION: 'last_name' initialized to ''
    last_name = Str('') 
\end{lstlisting}
\pause
\begin{lstlisting}
class Child(HasTraits):
    age = Int
    # VALIDATION: 'father' must be Parent instance
    father = Instance(Parent)
    # DELEGATION: 'last_name' delegated to father's 
    last_name = Delegate('father') 
    # NOTIFICATION: Method called when 'age' changes
    def _age_changed(self, old, new): 
        print 'Age changed from %s to %s ' % (old, new)
\end{lstlisting}
\end{frame}

\begin{frame}[fragile,plain]
  \frametitle{Traits Example}
\vspace*{-6pt}
\small
\begin{lstlisting}
In []: joe = Parent()
In []: joe.last_name = 'Johnson'
In []: moe = Child()
In []: moe.father = joe

In []: moe.last_name # Delegation
Out[]: "Johnson"

In []: moe.age = 10 # Notification
Age changed from 0 to 10

In []: moe.configure_traits() # Visualization
\end{lstlisting}
\end{frame}

\begin{frame}
  \frametitle{Traits}

  \begin{center}
  \alert{For more details, don't miss Eric's talk tomorrow!}
  \end{center}

\end{frame}

\begin{frame}
  \frametitle{Simple UI for the Mandelbrot set}
\begin{center}
    Goal: Build a simple UI for the Mandelbrot set example.
\end{center}
\end{frame}

\begin{frame}
    \frametitle{Requirements}

    \begin{itemize}
        \item Change the width and height
        \item Change the region
        \item Time the execution
    \end{itemize}
\end{frame}

\begin{frame}[fragile]
  \frametitle{Changes to the code}
  \alert{This is only so we can change the region}
  \begin{lstlisting}
def mandel_np(h, w, xmin=-2, xmax=0.8, 
              ymin=-1.4, ymax=1.4, 
              maxit=20):
    """Returns the 'Mandelbrot set'."""
    x, y = np.ogrid[xmin:xmax:w*1j, 
                    ymin:ymax:h*1j]
    # ... 
  \end{lstlisting}

All the implementations are in \typ{mbrt.pyx}
\end{frame}

\begin{frame}
    \frametitle{The steps involved}
    \begin{enumerate}
        \item A simple class (with traits) to manage the parameters we want
        \item Suitable callbacks to implement functionality
        \item Plot results
    \end{enumerate}
\end{frame}

\begin{frame}[fragile,plain]
  \frametitle{The code: part 1}
\small
\begin{lstlisting}

import time # for timing

import mbrt # our mandelbrot module

# Various trait types for the class.
from enthought.traits.api import (HasTraits, 
  Float, Range, Enum, Array, on_trait_change,
  Instance)

class Mandelbrot(HasTraits):
    # Traits ...

    # Methods ...
  \end{lstlisting}
\end{frame}

\begin{frame}[fragile,plain]
  \frametitle{The code: part 2}
\small
\begin{lstlisting}
class Mandelbrot(HasTraits):
    method = Enum('cython', 'python', 'numpy', 
                 desc='implementation to use')
\end{lstlisting}
\pause
\vspace*{-1em}
\begin{lstlisting}
    width = Range(10, 5000, 256, desc='width...')
\end{lstlisting}
\pause
\vspace*{-1em}
\begin{lstlisting}
    height = Range(10, 5000, 256, desc='...')
    iterations = Range(2, 100, 20)
    xmin = Range(-4., 4., -2.)
    xmax = Range(-4., 4., 0.8)
    ymin = Range(-4., 4., -1.4, desc='...')
    ymax = Range(-4., 4., 1.4, desc='...')

    runtime = Float(0.0, desc='execution time')
    # The result of the computation.
    result = Array()
  \end{lstlisting}
\end{frame}

\begin{frame}[fragile,plain]
  \frametitle{The code: part 3}
\small
\begin{lstlisting}
class Mandelbrot(HasTraits):
    # Traits ...
    @on_trait_change('method, width, height, '\
                     'xmin, xmax, ymin, '\
                     'ymax, iterations')
    def compute(self):
        """Calculate the Mandelbrot set"""
        methods = {'python': mbrt.mandel_py, 
                   'cython': mbrt.mandel_cy,
                   'numpy': mbrt.mandel_np}
        mandel = methods[self.method]
        t1 = time.time()
        result = mandel(self.height, ...)
        t2 = time.time()
        # Set the results.
        self.runtime = t2 - t1
        self.result = result
  \end{lstlisting}
\end{frame}

\begin{frame}[fragile,plain]
  \frametitle{The code: part 4}
\small
\begin{lstlisting}
if __name__ == '__main__':
    m = Mandelbrot()
    m.compute() # Compute it once.
    m.configure_traits()
# EOF
$ python mandel_ui0.py
  \end{lstlisting}
  \begin{center}
    \pgfimage[width=1.5in]{data/mandel_ui0}
  \end{center}
\end{frame}

\begin{frame}
    \frametitle{Issues}
    \begin{enumerate}
    \item Relatively simple
    \item Works
    \item Doesn't look great
    \end{enumerate}
    \begin{block}{What we need}
        \begin{enumerate}
            \item Prettier layout
            \item The actual result
        \end{enumerate}
    \end{block}
\end{frame}

\begin{frame}[fragile,plain]
    \frametitle{Fixing the layout: \typ{Views}}
    \vspace*{-1ex}
\small
\begin{lstlisting}
from enthought.traits.ui.api import (View,
  Item, Group)
\end{lstlisting}
\pause
\begin{lstlisting}
class Mandelbrot(HasTraits):
    # Traits ...
    view = View(Group(
                  Item(name='method'),
                  Item(name='runtime'),
                  Item(name='width'),
                  Item(name='height'),
                  Item(name='xmin'),
                  Item(name='xmax'),
                  Item(name='ymin'),
                  Item(name='ymax'),
                  Item(name='iterations'),
                  )
                )
    # Methods ...
  \end{lstlisting}
\end{frame}

\begin{frame}
    \frametitle{Plotting: using Chaco}
    \begin{itemize}
        \item Chaco is very powerful
        \item Could use Matplotlib but is a bit more work
    \end{itemize}
\end{frame}

\begin{frame}[fragile,plain]
  \frametitle{The code: chaco plotting}
\small
\begin{lstlisting}
from enthought.chaco.api import \
    ArrayPlotData, Plot, jet
from enthought.enable.component_editor \
    import ComponentEditor
    
class Mandelbrot(HasTraits):
    # ... 
    plotdata = Instance(ArrayPlotData)
    plot = Instance(Plot)
    view = View(Group(...
                  Item(name='iterations'),
                  Item(name='plot',
                       editor=ComponentEditor(), show_label=False)
                  ),
                  width=500, height=500,
                  resizable=True,
                  title="Mandelbrot set")
  \end{lstlisting}
\end{frame}

\begin{frame}[fragile,plain]
  \frametitle{The code: chaco plotting (ctd.)}
\small
\begin{lstlisting}
class Mandelbrot(HasTraits):
    # ... 
    def _result_changed(self, res):
        if self.plot is None:
            plotdata = ArrayPlotData(img = res)
            self.plotdata = plotdata
            # Create a Plot and associate it 
            # with the PlotData
            plot = Plot(plotdata)
            # Create an image plot in the Plot
            plot.img_plot("img", colormap=jet)
            self.plot = plot
        else:
            self.plotdata.set_data('img', res)
  \end{lstlisting}
\end{frame}

\begin{frame}[plain]
  \frametitle{The result}
  \begin{center}
    \pgfimage[width=2.5in]{data/mandel_ui}
  \end{center}
\end{frame}

\begin{frame}[plain]
  %\frametitle{A review of the code}
  \begin{center}
      \Huge
      \myemph{A Review of the Code}
  \end{center}
\end{frame}

\section{Mayavi}

\begin{frame}[fragile]
\frametitle{Basic 3D visualization}
\begin{itemize}
    \item Simple one liners for 3D visualization
\end{itemize}
  \begin{lstlisting}
$ ipython --pylab --wthread
# or
$ ipython --pylab --q4thread
  \end{lstlisting}
\begin{itemize}
    \item Must import mlab.
\end{itemize}
\begin{lstlisting}
In []: from enthought.mayavi import mlab
\end{lstlisting}
Ready to go!
\end{frame}

\begin{frame}[fragile]
    \frametitle{Simple examples}

    \myemph{\Large Try these}

    \begin{lstlisting}
In []: mlab.test_<TAB>

In []: mlab.test_points3d()
In []: mlab.clf()
In []: mlab.test_plot3d()
In []: mlab.clf()
In []: mlab.test_surf()
In []: mlab.test_surf??
    \end{lstlisting}
Explore the UI.
\end{frame}


\begin{frame}[fragile]
    \frametitle{\typ{mlab} plotting functions}
    \begin{columns}
        \column{0.25\textwidth}
        \myemph{\Large 0D data}
        \column{0.5\textwidth}
    \pgfimage[width=2in]{../MEDIA/m2/mlab/points3d_ex}
    \end{columns}

    \begin{lstlisting}
In []: t = linspace(0, 2*pi, 50)
In []: u = cos(t) * pi
In []: x, y, z = sin(u), cos(u), sin(t)
    \end{lstlisting}
    \emphbar{\PythonCode{In []: mlab.points3d(x, y, z)}}
\end{frame}

\begin{frame}[plain, fragile]
    \frametitle{Changing how things look}

    \begin{block}{Clearing the view}
    \PythonCode{>>> mlab.clf()}
    \end{block}

    \pause

    \begin{block}{IPython is your friend!}
    \PythonCode{>>> mlab.points3d?}

    \begin{itemize}

        \item Extra argument: Scalars

        \item Keyword arguments

        \item UI
    \end{itemize}
    \end{block}
\pause
    \begin{lstlisting}
In []: mlab.points3d(x, y, z, t, 
                     scale_mode='none')
    \end{lstlisting}

\end{frame}



\begin{frame}[fragile]
  \begin{columns}
        \column{0.25\textwidth}
        \myemph{\Large 1D data}
        \column{0.5\textwidth}
        \pgfimage[width=2.5in]{../MEDIA/m2/mlab/plot3d_ex}
  \end{columns}
  \emphbar{\PythonCode{In []: mlab.plot3d(x, y, z, t)}}

    Plots lines between the points
    
\end{frame}

\begin{frame}[fragile]
    \begin{columns}
        \column{0.25\textwidth}
        \myemph{\Large 2D data}
        \column{0.5\textwidth}
        \pgfimage[width=2in]{../MEDIA/m2/mlab/surf_ex}
    \end{columns}            
    \begin{lstlisting}
In []: x, y = mgrid[-3:3:100j,-3:3:100j]
In []: z = sin(x*x + y*y)
    \end{lstlisting}

    \emphbar{\PythonCode{In []: mlab.surf(x, y, z)}}

    \alert{Assumes the points are rectilinear}

\end{frame}

\begin{frame}[plain, fragile]
    \myemph{\Large 2D data: \texttt{mlab.mesh}}
    \vspace*{0.25in}

    \emphbar{\PythonCode{>>> mlab.mesh(x, y, z)}}

    \alert{Points needn't be regular}

    \vspace*{0.25in}
\begin{lstlisting}
>>> phi, theta = numpy.mgrid[0:pi:20j, 
...                         0:2*pi:20j]
>>> x = sin(phi)*cos(theta)
>>> y = sin(phi)*sin(theta)
>>> z = cos(phi)
>>> mlab.mesh(x, y, z, 
...           representation='wireframe')
\end{lstlisting}

\end{frame}

\begin{frame}[plain, fragile]

        \myemph{\Large 3D data}
    \vspace*{0.25in}

    \pgfimage[width=1.5in]{../MEDIA/m2/mlab/contour3d}\\
    
\begin{lstlisting}
>>> x, y, z = ogrid[-5:5:64j, 
...                -5:5:64j, 
...                -5:5:64j]
>>> mlab.contour3d(x*x*0.5 + y*y + 
                   z*z*2)
\end{lstlisting}
\end{frame}

\begin{frame}[plain, fragile]

    \myemph{\Large 3D vector data: \PythonCode{mlab.quiver3d}}
    \vspace*{0.25in}

    \pgfimage[width=2in]{../MEDIA/m2/mlab/quiver3d_ex}\\
    
\begin{lstlisting}
>>> mlab.test_quiver3d()
\end{lstlisting}

\emphbar{\PythonCode{obj = mlab.quiver3d(x, y, z, u, v, w)}}

\end{frame}

\begin{frame}[plain, fragile]
    \frametitle{Exercise: Lorenz equation}
    \begin{eqnarray*}
        \frac{d x}{dt} &=& s (y-x)\\
        \frac{d y}{d t} &=& rx -y -xz\\
        \frac{d z}{d t} &=& xy - bz\\
    \end{eqnarray*}
Let $s=10,r=28, b=8./3.$
\vspace*{0.2in}

    \structure{\Large Region of interest}
  \begin{lstlisting}
x, y, z = mgrid[-50:50:20j,-50:50:20j,
                -10:60:20j]
 \end{lstlisting}

 Use \PythonCode{mlab.quiver3d}

\end{frame}

\begin{frame}[plain,fragile]
    \frametitle{Solution}
  \begin{lstlisting}
def lorenz(x, y, z, s=10.,r=28., b=8./3.):
    u = s*(y-x)
    v = r*x -y - x*z
    w = x*y - b*z
    return u, v, w

x, y, z = mgrid[-50:50:20j,-50:50:20j,
                -10:60:20j]
u, v, w = lorenz(x, y, z)

mlab.quiver3d(x, y, z, u, v, w,
              scale_factor=0.01,
              mask_points=5)
mlab.show()
 \end{lstlisting}
\end{frame}

\begin{frame}[plain]
    {Issues and solutions}

    \Large 

    \begin{itemize}

        \item Basic visualization: not very useful

        \item Tweak parameters: \PythonCode{mask_points, scale_factor}

        \item Explore parameters on UI

        \item \PythonCode{mlab.flow} is a lot better!

    \end{itemize}

    \emphbar{\alert{Good} visualization involves work}
 
\end{frame}

\begin{frame}[plain]
    {Other utility functions}
    \Large
    \begin{itemize}
        \item \PythonCode{gcf}: get current figure
            \pause
        \item \PythonCode{savefig}, \PythonCode{figure}
            \pause
        \item \PythonCode{axes}, \PythonCode{outline}
            \pause
        \item \PythonCode{title}, \PythonCode{xlabel, ylabel, zlabel}
            \pause
        \item \PythonCode{colorbar}, \PythonCode{scalarbar},
            \PythonCode{vectorbar}
            \pause
        \item \PythonCode{show}: Standalone mlab scripts
            \pause
        \item Others, see UG
    \end{itemize}

\end{frame}

\section{Sage}

\begin{frame}[plain]
    \frametitle{Sage}
  \begin{center}    
    \pgfimage[width=3.5in,interpolate=true]{../MEDIA/python/sage}
  \end{center}
\end{frame}

\end{document}

