%%%%%%%%%%%%%%%%%%%%%%%%%%%%%%%%%%%%%%%%%%%%%%%%%%%%%%%%%%%%%%%%%%%%%%%%%%%%%%%%
%Tutorial slides on Python.
%
% Author: FOSSEE 
% Copyright (c) 2009-2011, FOSSEE, IIT Bombay
%%%%%%%%%%%%%%%%%%%%%%%%%%%%%%%%%%%%%%%%%%%%%%%%%%%%%%%%%%%%%%%%%%%%%%%%%%%%%%%%

\documentclass[14pt,compress]{beamer}

% Modified from: generic-ornate-15min-45min.de.tex
\mode<presentation>
{
  \usetheme{Warsaw}
  \useoutertheme{infolines}
  \setbeamercovered{transparent}
}

\usepackage[english]{babel}
\usepackage[latin1]{inputenc}
%\usepackage{times}
\usepackage[T1]{fontenc}
\usepackage{pgf}

% Taken from Fernando's slides.
\usepackage{ae,aecompl}
\usepackage{mathpazo,courier,euler}
\usepackage[scaled=.95]{helvet}

\definecolor{darkgreen}{rgb}{0,0.5,0}

\usepackage{listings}
\lstset{language=Python,
    basicstyle=\ttfamily\bfseries,
    commentstyle=\color{red}\itshape,
  stringstyle=\color{darkgreen},
  showstringspaces=false,
  keywordstyle=\color{blue}\bfseries}

%%%%%%%%%%%%%%%%%%%%%%%%%%%%%%%%%%%%%%%%%%%%%%%%%%%%%%%%%%%%%%%%%%%%%%
% Macros
\setbeamercolor{emphbar}{bg=blue!20, fg=black}
\newcommand{\emphbar}[1]
{\begin{beamercolorbox}[rounded=true]{emphbar} 
      {#1}
 \end{beamercolorbox}
}
\newcounter{time}
\setcounter{time}{0}
\newcommand{\inctime}[1]{\addtocounter{time}{#1}{\tiny \thetime\ m}}

\newcommand{\typ}[1]{\lstinline{#1}}

\newcommand{\kwrd}[1]{ \texttt{\textbf{\color{blue}{#1}}}  }

%%% This is from Fernando's setup.
% \usepackage{color}
% \definecolor{orange}{cmyk}{0,0.4,0.8,0.2}
% % Use and configure listings package for nicely formatted code
% \usepackage{listings}
% \lstset{
%    language=Python,
%    basicstyle=\small\ttfamily,
%    commentstyle=\ttfamily\color{blue},
%    stringstyle=\ttfamily\color{orange},
%    showstringspaces=false,
%    breaklines=true,
%    postbreak = \space\dots
% }

%%%%%%%%%%%%%%%%%%%%%%%%%%%%%%%%%%%%%%%%%%%%%%%%%%%%%%%%%%%%%%%%%%%%%%
% Title page
\title[Advanced Sci Comp.]{Advanced Scientific Computing with
Python}
\subtitle{Introduction, NumPy, Virtualenv}

\author[FOSSEE group] {FOSSEE}

\institute[IIT Bombay] {Department of Aerospace Engineering\\IIT Bombay}
\date[] {PyCon Asia-Pacific,\\
Singapore\\
June 9, 2011
}
%%%%%%%%%%%%%%%%%%%%%%%%%%%%%%%%%%%%%%%%%%%%%%%%%%%%%%%%%%%%%%%%%%%%%%

%\pgfdeclareimage[height=0.75cm]{iitmlogo}{iitmlogo}
%\logo{\pgfuseimage{iitmlogo}}


%% Delete this, if you do not want the table of contents to pop up at
%% the beginning of each subsection:
\AtBeginSubsection[]
{
  \begin{frame}<beamer>
    \frametitle{Outline}
    \tableofcontents[currentsection,currentsubsection]
  \end{frame}
}

\AtBeginSection[]
{
  \begin{frame}<beamer>
    \frametitle{Outline}
    \tableofcontents[currentsection,currentsubsection]
  \end{frame}
}

% If you wish to uncover everything in a step-wise fashion, uncomment
% the following command: 
%\beamerdefaultoverlayspecification{<+->}

%%\includeonlyframes{current,current1,current2,current3,current4,current5,current6}

%%%%%%%%%%%%%%%%%%%%%%%%%%%%%%%%%%%%%%%%%%%%%%%%%%%%%%%%%%%%%%%%%%%%%%
% DOCUMENT STARTS
\begin{document}

\begin{frame}
  \maketitle
\end{frame}

%% \begin{frame}
%%   \frametitle{Outline}
%%   \tableofcontents
%%   % You might wish to add the option [pausesections]
%% \end{frame}

\begin{frame}
    \frametitle{Acknowledgement}
    \Large
    \begin{center}
        \alert{FOSSEE group (\url{fossee.in})} \\
        based at\\ 
        \alert{IIT Bombay}\\
        and funded by\\
        The National Mission on Education through ICT, \\
        \alert{Ministry of HRD, India}
    \end{center}
\end{frame}

\section{Checklist}
\begin{frame}
\frametitle{Checklist}
  \begin{enumerate}
    \item \alert{FIXME update when this session is done!}
    \item Data files: 
      \begin{itemize}
      \item \typ{anag.txt}
      \item \typ{holmes.txt}
      \item \typ{pendulum.txt}
      \item \typ{pos.txt}
      \item \typ{sslc1.txt}
      \end{itemize}
    \item Python scripts: 
      \begin{itemize}
      \item \typ{sslc_allreg.py}
      \item \typ{sslc_science.py}
      \end{itemize}
    \item Images
      \begin{itemize}
      \item \typ{lena.png}
      \item \typ{smoothing.gif}
      \end{itemize}
  \end{enumerate}
\end{frame}

\begin{frame}
  \frametitle{About the Tutorial}
  \begin{block}{Intended Audience}
  \begin{itemize}
       \item Engg., Mathematics and Science researchers with a
           good knowledge of Python, some OOP and knowledge of the material
           covered in the introductory tutorial.
  \end{itemize}
  \end{block}  

  \begin{block}{Goal: Successful participants will be able to}
    \begin{itemize}
      \item Start using Python as a tool for computational research

      \item Use some of the more advanced tools for their own research

    \end{itemize}
  \end{block}
\end{frame}

\section{Introduction}

\begin{frame}
    \frametitle{Tools available}
    \begin{itemize}
        \item IPython: the shell
        \item Numpy/Scipy: array and numerics
        \item Matplotlib/chaco: 2d plotting
        \item Mayavi: 3d plotting
        \item Enthought Tool Suite (ETS): application building
        \item Sympy: symbolic math 
        \item Cython: Python $\rightarrow$ C compiler 
        \item mpi4py, PyOpenCL, PyCUDA etc. 
            
            \vspace*{0.2in}
        \item Sage: very powerful symbolic and numeric tools
        \item Other specialized tools 

    \end{itemize}
\end{frame}

\section{IPython}

\begin{frame}
  \frametitle{IPython}
  \begin{itemize}
  \item Recommended interpreter, IPython:
    \url{http://ipython.scipy.org}
  \item Better than default Python shell
  \item Tab completion
  \item Easier object introspection
  \item Shell access!
  \item Command system
  \item Supports history and logging
  \item Embeddable
  \item Macros
  \item Custom interpreter
  \item Advanced features 
  \end{itemize}
\end{frame}

\begin{frame}[fragile]
  \frametitle{Basic IPython features}
  \begin{itemize}
  \item \verb+[--wthread|--gthread|--qthread]+
  \item \verb+--pylab+: Support for matplotlib
  \item \verb+object?+
  \item \verb+object??+
  \item \verb+%pdb+: pops up pdb on errors
  \item History: \texttt{<UpArrow>}
  \item Search: \texttt{<Ctrl-r> string}
  \item End of history: \texttt{Esc >}
  \item \verb+%run [options] file[.py]+
  \end{itemize}
\end{frame}

\begin{frame}[fragile]
  \frametitle{More IPython features}
  \begin{itemize}
  \item Input and output caching: \verb+In+, \verb+Out+
    \item \verb+%hist+ 
    \item Log the session using \verb+%logstart+, \verb+%logon+ and
    \verb+%logoff+
    \item \verb+%run -d:+ debug script with pdb
    \item \verb+%run -t+: time the script
    \item \verb+%run -p+: Profile the script
  \item \verb+%macro+ 
  \end{itemize}
\end{frame}

\begin{frame}[fragile]
  \frametitle{More IPython features \ldots}
  \begin{itemize}
  \item \verb+%edit [options] [args]+
  \item \verb+%cd, pushd, popd, dhist+ directory
    \item \verb+!command+: shell command
    \item \verb+files = %sx ls+ or \verb+files = !ls+
    \item \verb+%sx+ is quiet
    \item \verb+!ls $files+ passes the \verb+files+
    \item \verb+%alias alias_name cmd+
  \end{itemize}
\end{frame}

\begin{frame}[fragile]
  \frametitle{More IPython features \ldots}
  \begin{itemize}
  \item \verb+;+ suppresses output
  \item \verb+%bookmark+
  \item \verb+%who, %whos+: print information on variables
  \item \verb+%save+: save lines to a file
  \item \verb+%time statement+
  \item \verb+%timeit statement+
  \item Profiles: \verb+math, scipy, numeric, pysh+ etc.
  \item \verb+%magic+: \alert{Show help on all magics}
  \end{itemize}
\end{frame}

\section{NumPy arrays}

\newcommand{\num}{\texttt{numpy}}


\begin{frame}[fragile]
  \frametitle{The \num\ module}
    \begin{itemize}
    \item Efficient, powerful array type
    \item Abstracts out standard operations on arrays
    \item COnvenience functions
    \item \typ{ipython --pylab} imports part of numpy
    \item Without the Pylab mode do:
    \end{itemize}
    \begin{lstlisting}
In []: import numpy

In []: from numpy import *
    \end{lstlisting}
\end{frame}

\begin{frame}
  \frametitle{\num\ arrays}
  \begin{itemize}
  \item Fixed size (\typ{arr.size})
  \item Same type (\typ{arr.dtype})
  \item Arbitrary dimensionality: \typ{arr.shape}
  \item \typ{shape}: extent (size) along each dimension
  \item \typ{arr.itemsize}: number of bytes per element
  \item \alert{Note:} \typ{shape} can change so long as the \typ{size}
      is constant
  \item Indices start from 0
  \item Negative indices do the right thing.
  \end{itemize}
\end{frame}

\begin{frame}[fragile]
  \frametitle{\num\ arrays}
\begin{lstlisting}
In []: a = array([1,2,3,4])
In []: b = array([2,3,4,5])

In []: print a[0], a[-1]
Out[]: (1, 4)

In []: a[0] = -1
In []: a[0] = 1
\end{lstlisting}
Operations are elementwise
\end{frame}

\begin{frame}[fragile]
  \frametitle{Simple operations}
\begin{lstlisting}
In []: a + b  
Out[]: array([3, 5, 7, 9])
In []: a*b
Out[]: array([2, 6, 12, 20])
In []: a/b
Out[]: array([0, 0, 0, 0])
\end{lstlisting}
Operations are elementwise, types matter.
\end{frame}

\begin{frame}[fragile]
  \frametitle{Data type matters}
  Try again with this:
\begin{lstlisting}
In []: a = array([1.,2,3,4])
In []: a/b 
\end{lstlisting}
\end{frame}

\begin{frame}[fragile]
  \frametitle{Examples}
\noindent \typ{pi} and \typ{e} are defined.
\begin{lstlisting}
In []: x = linspace(0.0, 10.0, 200)
In []: x *= 2*pi/10
# apply functions to array.
In []: y = sin(x)
In []: y = cos(x)
In []: x[0] = -1
In []: print x[0], x[-1]
-1.0 10.0
\end{lstlisting}
\end{frame}

\begin{frame}[fragile]
    \frametitle{\typ{size, shape, rank} etc.}
\vspace*{-8pt}
\begin{lstlisting}
In []: x = array([1., 2, 3, 4])
In []: size(x)
Out[]: 4
In []: x.dtype
dtype('float64')
In []: x.shape
Out[] (4,)
In []: rank(x)
Out[]: 1
In []: x.itemsize
Out[]: 8
\end{lstlisting}
\end{frame}


\begin{frame}[fragile]
  \frametitle{Multi-dimensional arrays}
\begin{lstlisting}
In []: a = array([[ 0, 1, 2, 3],
  ...:            [10,11,12,13]])
In []: a.shape # (rows, columns)
Out[]: (2, 4)

In []: a[1,3] 
Out[]: 13

In []: a[1,3] = -1
In []: a[1] # The second row
array([10,11,12,-1])
In []: a[1] = 0 # Entire row to zero.
\end{lstlisting}

\end{frame}

\begin{frame}[fragile]
  \frametitle{Slicing arrays}
  \vspace*{-0.2in}
\begin{lstlisting}
In []: a = array([[1,2,3], [4,5,6], 
  ...:            [7,8,9]])
In []: a[0,1:3]
Out[]: array([2, 3])
In []: a[1:,1:]
Out[]: array([[5, 6],
              [8, 9]])
In []: a[:,2]
Out[]: array([3, 6, 9])
In []: a[0::2,0::2] # Striding...
Out[]: array([[1, 3],
              [7, 9]])
# Slices refer to the same memory!
\end{lstlisting}
\end{frame}

\begin{frame}[fragile]
  \frametitle{Array creation functions}
  \begin{itemize}
  \item \typ{array(object)}
  \item \typ{linspace(start, stop, num=50)}
  \item \typ{ones(shape)}
  \item \typ{zeros((d1,...,dn))}
  \item \typ{empty((d1,...,dn))}
  \item \typ{identity(n)}
  \item \typ{ones\_like(x)}, \typ{zeros\_like(x)}, \typ{empty\_like(x)}
  \end{itemize}
  May pass an optional \typ{dtype=} keyword argument

  For more dtypes see: \typ{numpy.typeDict}
\end{frame}

\begin{frame}[fragile]
  \frametitle{Creation examples}
  \vspace*{-0.25in}
\begin{lstlisting}
In []: a = array([1,2,3], dtype=float)
In []: ones( (2, 3) )
Out[]: array([[ 1.,  1.,  1.],
              [ 1.,  1.,  1.]])
In []: identity(3)
Out[]: array([[ 1.,  0.,  0.],
              [ 0.,  1.,  0.],
              [ 0.,  0.,  1.]])
In []: ones_like(a)
Out[]: array([ 1.,  1.,  1.,  1.])
\end{lstlisting}
\end{frame}

\begin{frame}[fragile]
  \frametitle{Array math}
  \begin{itemize}
  \item Basic \alert{elementwise} math (given two arrays \typ{a, b}):
    \begin{itemize}
        \item \typ{a + b} $\rightarrow$ \typ{add(a, b)} 
        \item \typ{a - b}, $\rightarrow$ \typ{subtract(a, b)} 
        \item \typ{a * b}, $\rightarrow$ \typ{multiply(a, b)} 
        \item \typ{a / b}, $\rightarrow$ \typ{divide(a, b)} 
        \item \typ{a \% b}, $\rightarrow$ \typ{remainder(a, b)} 
        \item \typ{a ** b}, $\rightarrow$ \typ{power(a, b)}
    \end{itemize}
  \item Inplace operators: \typ{a += b}, or \typ{add(a, b,
      a)}
  \item Logical operations: \typ{==, !=, <, >}, etc.
  \item \typ{sin(x), arcsin(x), sinh(x)},
      \typ{exp(x), sqrt(x)} etc.
  \item \typ{sum(x, axis=0), product(x, axis=0)}
  \item \typ{dot(a, b)}
  \end{itemize}
\end{frame}



\section{SciPy}
\begin{frame}[plain]
  \frametitle{SciPy}
  \begin{itemize}
  \item Provides:
    \begin{itemize}
    \item Linear algebra
    \item Numerical integration
    \item Fourier transforms
    \item Signal processing
    \item Special functions
    \item Statistics
    \item Optimization
    \item Image processing
    \item ODE solvers
    \end{itemize}
  \item Uses LAPACK, QUADPACK, ODEPACK, FFTPACK etc. from netlib
  \end{itemize}
\end{frame}

\begin{frame}[fragile,plain]
  \frametitle{Matplotlib plots}
  \begin{columns}
    \column{0.5\textwidth}
    \hspace*{-0.5in}
    \pgfimage[height=2in, interpolate=true]{../MEDIA/python/xyplot}
    \column{0.45\textwidth}
    \begin{block}{Example code}
    \tiny
\begin{lstlisting}
t1 = arange(0.0, 5.0, 0.1)
t2 = arange(0.0, 5.0, 0.02)
t3 = arange(0.0, 2.0, 0.01)
subplot(211)
plot(t1, cos(2*pi*t1)*exp(-t1), 'bo', 
     t2, cos(2*pi*t2)*exp(-t2), 'k')
grid(True)
title('A tale of 2 subplots')
ylabel('Damped')
subplot(212)
plot(t3, cos(2*pi*t3), 'r--')
grid(True)
xlabel('time (s)')
ylabel('Undamped')
\end{lstlisting}
    \end{block}
  \end{columns}
\end{frame}

\begin{frame}[plain]
    \begin{center}
  \pgfimage[height=1.5in, interpolate=true]{../MEDIA/python/errorbar}  
  \pgfimage[height=1.5in, interpolate=true]{../MEDIA/python/log}  
    \end{center}
\end{frame}
\begin{frame}[plain]
    \begin{center}
  \pgfimage[height=1.75in, interpolate=true]{../MEDIA/python/barchart}  
  \pgfimage[height=1.75in, interpolate=true]{../MEDIA/python/piechart}  
    \end{center}
\end{frame}

\begin{frame}[plain]
    \begin{center}
  \pgfimage[height=1.75in, interpolate=true]{../MEDIA/python/scatter}  
  \pgfimage[height=1.75in, interpolate=true]{../MEDIA/python/histogram}  
    \end{center}
\end{frame}

\begin{frame}[plain]
    \begin{center}
  \pgfimage[height=2in, interpolate=true]{../MEDIA/python/polar}  
  \pgfimage[height=1.75in, interpolate=true]{../MEDIA/python/quiver}  
    \end{center}
\end{frame}
\begin{frame}[plain]
    \begin{center}
  \pgfimage[height=2.5in, interpolate=true]{../MEDIA/python/plotmap}  
    \end{center}
\end{frame}

\begin{frame}[plain]
    \frametitle{IPython + SciPy + Matplotlib}
  \begin{center}
  \pgfimage[height=1.75in, interpolate=true]{../MEDIA/python/contour}  
  \pgfimage[height=2.75in, interpolate=true]{../MEDIA/python/scipy_screenie}  
  \end{center}
\end{frame}

\end{document}

